% !TEX root = ../main.tex

%%%%%%%%%%%%%%%%%%%%%%%%%%% 其他设置(勿动) %%%%%%%%%%%%%%%%%%%%%%%%%%%

\ifphd
  \degree{博士}
  \englishdegree{Ph.D}
  \englishdegreea{Doctoral}
  \englishdegreeb{Doctor}
\else
  \degree{硕士}
  \englishdegree{Master}
  \englishdegreea{Masteral}
  \englishdegreeb{Master}
\fi

%% 为了方便输入特殊符号, 添加如下新命令:
\medmuskip=2mu        %水平间距调整: 二元运算符 "+, -, <"
\thickmuskip=3mu      %水平间距调整: 关系符号调整 "="
% \abovedisplayskip=0pt
% \belowdisplayskip=0pt
% \abovedisplayshortskip=0pt
% \belowdisplayshortskip=0pt

%\ctexset{section={format+={\flushleft}}}  % 小节标题靠左对齐

%% 页面大小, 不要设置左右边距, 否则将出现页眉和正文无法对齐
%% 尽管研究生院要求上下左右边距均为2.5cm, 但没有考虑预留奇偶页边距用于装订成册 
\geometry{top=3.5cm,bottom=3.5cm}

%%%%%%%%%%%%%%%%%%%%%%%%%%%%%%%%%%%%%%%%%%%%%%%%%%%%%%%%%%%%%%%%%%

% 常用符号与自定义命令
\def\ZZ{{\mathbb Z}}
\def\NN{{\mathbb N}}
\def\RR{{\mathbb R}}
\def\CC{{\mathbb C}}
\def\QQ{{\mathbb Q}}
\def\EE{{\mathbb E}}
\def\FF{{\mathbb F}}
\def\Fp{\mathbb{F}_{p}}
\def\Fq{\mathbb{F}_{q}}
\def\Zp{\mathbb{Z}_{p}}
\def\Zq{\mathbb{Z}_{q}}
\def\Zk{\mathbb{Z}_{2^k}}
\def\Zl{\mathbb{Z}_{2^\ell}}

%\usepackage{pifont}
\newcommand{\cmark}{\ding{51}} % 打勾
\newcommand{\xmark}{\ding{55}} % 打叉

%% for algorithm
\floatname{algorithm}{算法}
\renewcommand{\algorithmicrequire}{\textbf{输入:}}
\renewcommand{\algorithmicensure}{\textbf{输出:}}

% Protocol/Functionality
\tcbuselibrary{skins,breakable}
\newtcolorbox[auto counter]{QandA}[2][]{%
  enhanced,
  title        = {问题\thetcbcounter. {#2}},
  attach boxed title to top left={xshift=+3mm,yshift*=-3mm},
  colback      = black!5,
  colframe     = black!35,
  fonttitle    = \bfseries,
  colbacktitle = black!15!white,
  arc          = 0mm,
  coltitle     = black,
  #1
}
